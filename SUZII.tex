\documentclass[12pt]{amsart}

\usepackage{amsfonts,amsthm,amsmath,amssymb,amscd}

%(4.4) ---  (\ref{3093006N14p207a})
%\beq\label{4a100a}
%\log\l-\sup(\phi)-(k-1)\log d\ge \log\l-\sup(\phi)+\ka_f - k\log d>\th.
%\eeq


% "American" setting:
%\setlength{\topmargin}{0in} \setlength{\oddsidemargin}{0in}
%\setlength{\evensidemargin}{0in} \setlength{\textwidth}{6.4in}
%\setlength{\textheight}{8.6in}

%\usepackage{showkeys}
%\usepackage{diagrams}
%\usepackage{amsmath, amstext, amsthm, amsfonts, latexsym, color}
\usepackage{dsfont}
\usepackage{upref}
\usepackage[textures,backref,bookmarks=false]{hyperref}
\usepackage{mathrsfs}


% "French" setting:
\setlength{\topmargin}{1.5cm}
\evensidemargin 0.4truein\oddsidemargin 0.4truein
\textheight7.9truein
\textwidth5.9truein


\numberwithin{equation}{section}

%%%%%%%%%%%%%%%%%%%%%%%%%%%%%
% Parabolic, fractal measures
%
%%%%%%%%%%%%%%%%%%%%%%%%%%%%%
%\input epsf

\setlength{\itemsep}{0in}\newcommand{\lab}{\label}
\newcommand{\labeq}[1]{  \be \label{#1}  }
\newcommand{\labea}[1]{  \bea \label{#1}  }
\newcommand{\ben}{\begin{enumerate}}
\newcommand{\een}{\end{enumerate}}
\newcommand{\bm}{\boldmath}
\newcommand{\Bm}{\Boldmath}
\newcommand{\bea}{\begin{eqnarray}}
\newcommand{\ba}{\begin{array}}
\newcommand{\bean}{\begin{eqnarray*}}
\newcommand{\ea}{\end{array}}
\newcommand{\eea}{\end{eqnarray}}
\newcommand{\eean}{\end{eqnarray*}}
\newcommand{\beq}{\begin{equation}}
\newcommand{\eeq}{\end{equation}}
\newcommand{\bthm}{\begin{thm}}
\newcommand{\ethm}{\end{thm}}
\newcommand{\blem}{\begin{lem}}
\newcommand{\elem}{\end{lem}}
\newcommand{\bprop}{\begin{prop}}
\newcommand{\eprop}{\end{prop}}
\newcommand{\bcor}{\begin{cor}}
\newcommand{\ecor}{\end{cor}}
\newcommand{\bdfn}{\begin{dfn}}
\newcommand{\edfn}{\end{dfn}}
\newcommand{\brem}{\begin{rem}}
\newcommand{\erem}{\end{rem}}
\newcommand{\bpf}{\begin{proof}}
\newcommand{\epf}{\end{proof}}
\newcommand{\bfact}{\begin{fact}}
\newcommand{\efact}{\end{fact}}


\newcommand{\lb}{\linebreak}
\newcommand{\nlb}{\nolinebreak}
\newcommand{\nl}{\newline}
\newcommand{\hs}{\hspace}
\newcommand{\vs}{\vspace}
\alph{enumii} \roman{enumiii}

\newtheorem{thm}{Theorem}[section]
\newtheorem{prop}[thm]{Proposition}
\newtheorem{lem}[thm]{Lemma}
\newtheorem{sublem}[thm]{Sublemma}
\newtheorem{cor}[thm]{Corollary}
\newtheorem{dfn}[thm]{Definition}
\newtheorem{rem}[thm]{Remark}
\newtheorem{fact}[thm]{Fact}
\newtheorem{as}[thm]{Assumption}
%********************** Letters "cal" :  \cXXX   *******************************
\def\cA{\mathcal A}             \def\cB{\mathcal B}       \def\cC{\mathcal C}
\def\cH{\mathcal H}             \def\cF{\mathcal F}       \def\cI{\mathcal I}
\def\cL{{\mathcal L}}           \def\cM{\mathcal M}       \def\cP{{\mathcal P}}
\def\cU{\mathcal U}             \def\cV{\mathcal V}       \def\cJ{\mathcal J}
\def\cS{\mathcal S}             \def\cD{\mathcal D}
\def\cW{\mathcal W}             \def\cY{\mathcal Y}



\def\endpf{\qed}
\def\ms{\medskip}
\def\N{{\mathbb N}}            \def\Z{{\mathbb Z}}      \def\R{{\mathbb R}}
\def\C{{\mathbb C}}            \def\T{{\mathbb T}}         \def\oc{\hat \C}
\def\Q{{\mathbb Q}}
\def\1{1\!\!1}
\def\and{\text{ and }}        \def\for{\text{ for }}
\def\tif{\text{ if }}         \def\then{\text{ then }}
\def\Cap{\text{Cap}}          \def\Con{\text{Con}} \def\Per{\text{{\rm Per}}}
\def\Comp{\text{Comp}}        \def\diam{\text{\rm {diam}}}
\def\dist{\text{{\rm dist}}}  \def\Dist{\text{{\rm Dist}}}
\def\Crit{\text{Crit}}        \def\vol{\text{{\rm vol}}}
\def\Sing{\text{Sing}}        \def\conv{\text{{\rm conv}}}
\def\Fin{{\mathcal F}in}      \def\Mod{\text{{\rm Mod}}}
\def\F{{\mathcal F}}          \def\Jac{\text{{\rm Jac}}}
\def\h{{\text h}}
\def\hmu{\h_\mu}           \def\htop{{\text h_{\text{top}}}}
\def\H{\text{{\rm H}}}     \def\HD{\text{{\rm HD}}}   \def\DD{\text{DD}}
\def\BD{\text{{\rm BD}}}   \def\PD{\text{PD}}         \def\PC{\text{{\rm PC}}}
\def\re{\text{{\rm Re}}}    \def\im{\text{{\rm Im}}}  \def\PG{\text{{\rm PG}}}
\def\Int{\text{{\rm Int}}} \def\St{\text{{\rm St}}} \def\Area{\text{{\rm Area}}}
%\def\ep{\text{e}}
\def\CD{\text{CD}}         \def\P{\text{{\rm P}}}     \def\Id{\text{{\rm Id}}}
\def\Hyp{\text{{\rm Hyp}}} \def\sign{\text{{\rm sgn}}}
\def\A{\mathcal A}             \def\Ba{\mathcal B}       \def\Ca{\mathcal C}
\def\Ha{\mathcal H}
\def\L{{\mathcal L}}           \def\M{\mathcal M}        \def\Pa{{\mathcal P}}
\def\U{\mathcal U}             \def\V{\mathcal V}
\def\W{\mathcal W}             \def\Pk{{\mathbb P}^k}
\def\a{\alpha}                \def\b{\beta}             \def\d{\delta}
\def\De{\Delta}               \def\e{\varepsilon}          \def\f{\phi}
\def\g{\gamma}                \def\Ga{\Gamma}           \def\l{\lambda}
\def\La{\Lambda}              \def\om{\omega}           \def\Om{\Omega}
\def\Sg{\Sigma}               \def\sg{\sigma}
\def\Th{\Theta}               \def\th{\theta}           \def\vth{\vartheta}
\def\ka{\kappa}
\def\Ka{\Kappa}

\def\bi{\bigcap}              \def\bu{\bigcup}
\def\({\bigl(}                \def\){\bigr)}
\def\lt{\left}                \def\rt{\right}
\def\bv{\bigvee}
\def\ld{\ldots}               \def\bd{\partial}         \def\^{\tilde}
\def\club{\hfill{$\clubsuit$}}\def\proot{\root p\of}
\def\tm{\widetilde{\mu}}      \def\du{\bigoplus}
\def\tn{\widetilde{\nu}}
\def\es{\emptyset}            \def\sms{\setminus}
\def\sbt{\subset}             \def\spt{\supset}
\def\gek{\succeq}             \def\lek{\preceq}
\def\eqv{\Leftrightarrow}     \def\llr{\Longleftrightarrow}
\def\lr{\Longrightarrow}      \def\imp{\Rightarrow}
\def\comp{\asymp}
\def\upto{\nearrow}           \def\downto{\searrow}
\def\sp{\medskip}             \def\fr{\noindent}        \def\nl{\newline}
\def\ov{\overline}            \def\un{\underline}
\def\ess{{\rm ess}}
%\def\ni{\noindent}
\def\cl{\text{cl}}
%\def\QED{\hfill \Box}
\def\om{\omega}
\def\bt{{\bf t}}
\def\Bu{{\bf u}}
\def\tr{t}
\def\bo{{\bf 0}}
\def\nut{\nu_\lla^{\scriptscriptstyle 1/2}}
\def\arg{\text{arg}}
\def\Arg{\text{Arg}}
\def\re{\text{{\rm Re}}}
\def\gr{\nabla}
\def\supp{\text{{\rm supp}}}
\def\endpf{{\hfill $\square$}}
\def\Fa{\mathcal F}
\def\Gal{\mathcal G}
%\def\1{1\!\!1}
\def\D{{\mathbb D}}
%%%%%%%%%%%%%%%%%%%%%%%%%%%%%%%%%%%%%%%%%%%%%%%%%%%
\newcommand{\mint}{ \hspace{0.1cm} {\bf -} \hspace{-0.41cm} \int  }

\newcommand{\amsc}{{\mathbb C}} \newcommand{\amsr}{{\mathbb R}}
\newcommand{\amsz}{{\mathbb Z}}
\newcommand{\amsn}{{\mathbb N}} \newcommand{\amsq}{{\mathbb Q}}
\newcommand{\amsd}{{\mathbb D}}
\newcommand{\amss}{{\mathbb S}} \newcommand{\amsb}{{\mathbb B}}
\newcommand{\amsh}{{\mathbb H}}
\newcommand{\amst}{{\mathbb T}} \newcommand{\amsp}{{\mathbb P}}

\newcommand{\rr}{{\mathcal R}}
\newcommand{\cbar}{\hat{{\mathbb C}} }

\newcommand{\lam}{\lambda}
\newcommand{\ep}{\varepsilon}
\newcommand{\ph}{\varphi}
\newcommand{\al}{\alpha}
\newcommand{\ga}{\gamma}

%\newcommand{\jul}{{\mathcal J}_f}
\newcommand{\jul}{J(f)}
\newcommand{\fat}{{\mathcal F}_f}
\newcommand{\sing}{sing(f^{-1})}
\newcommand{\conical}{\Lambda _c}
\newcommand{\radii}{{\mathcal R}_z}
\newcommand{\exc}{{\mathcal E}_f}
\newcommand{\cri}{{\mathcal C}_f}
\newcommand{\post}{{\mathcal P}_f}
\newcommand{\measure}{C^\al _{\mathcal F}}


% Perron-Frobenius operator:
%\newcommand{\pf}{\mathcal{L}_\ph}
\newcommand{\pft}{{\mathcal{L}}_t}
\newcommand{\npft}{{\hat{\mathcal{L}}_t}}
\newcommand{\npf}{\tilde{\mathcal{L}}_\ph}
\newcommand{\good}{{\mathcal{G}}_\ph}
\newcommand{\bad}{{\mathcal{B}}_\ph}

\newcommand{\pftl}{{\mathcal{L}_{t,\l }}}
\newcommand{\pftolo}{{\mathcal{L}_{t_0 ,\l _0 }}}

\newcommand{\wei}{{\mathcal P}_\Lambda}
%******** derivees partielles complexes *************************
\newcommand{\pa}{\partial } \newcommand{\pab}{\overline{\partial }}
%************************ Theorems etc.  *********************
%***********************************************
%************************ Dfnts for this particular document:  *********************
\newcommand{\rad}{{\mathcal J}_r(f)}
\def\fo{f_{\l^0}}
\def\julo{{\mathcal J}(\fo)}
\def\jull{{\mathcal J}(f_\l)}
\newcommand{\radl}{{\mathcal J}_r(f_\l)}
\def\Jgg{{\mathcal J}(g_\g)}
\def\Jgo{{\mathcal J}(g_{\l^0})}
\def\jo{{\mathcal J}_0}
\def\J{\mathcal J}

\newcommand{\Jl}{{\mathcal J}_r(f_\l)}
\newcommand{\JPl}{{\mathcal J}_r(f\circ P_\l)}


%************************************************************************
\begin{document}
%***********************************************

\title[]
{ \bf\large {\Large S}tochastics and Thermodynamics \\ \ \\ for \\ \ \\
  Equilibrium Measures \\ of \\ Holomorphic 
  Endomorphisms \\ on \\ Complex Projective Spaces \\ \ \ }
\date{\today}
% % author information %

\author[\sc Michal SZOSTAKIEWICZ]{\sc Michal Szostakiewicz}
\address{Anna Zdunik, Institute of Mathematics, Warsaw University,
ul. Banacha 2, 02-097 Warszawa, Poland}
\email{M.Szostakiewicz@mimuw.edu.pl}
\author[\sc Mariusz URBA\'NSKI]{\sc Mariusz URBA\'NSKI}
\address{Mariusz Urba\'nski, Department of Mathematics,
 University of North Texas, Denton, TX 76203-1430, USA}
\email{urbanski\@unt.edu\newline \hspace*{0.3cm} Web:
www.math.unt.edu/$\sim$urbanski}
\author[\sc Anna ZDUNIK]{\sc Anna Zdunik}
\address{Anna Zdunik, Institute of Mathematics, Warsaw University,
ul. Banacha 2, 02-097 Warszawa, Poland}
\email{A.Zdunik\@@mimuw.edu.pl}

% dedication %
%\dedicatory{}
%
% AMS information %
\thanks{The research of the second author supported in part by the
NSF Grant DMS 1001874. The research of the first and third
author supported in 
part by the Polish MNiSW Grants N N201 0234 33 and N N201 607940}
\keywords{
Holomorphic dynamics, complex projective spaces,
Julia sets, topological pressure, equilibrium states, H\"older
continuous potentials, Perron-Frobenius operators, exponential decay
of correlations, Central Limit Theorem, the Law of Iterated Logarithm 
}
\subjclass{Primary: 37D35; Secondary: 32H50, 37C40, 28D05}

\begin{abstract}
It was proved in \cite{uzpk} that for every  holomorphic endomorphism
$f:\Pk\to\Pk$ of a complex 
projective space $\Pk$,$k\ge 1$, there exists a positive number
$\ka_f>0$ such that if $\phi:J\to\R$ is a H\"older continuous
function with $\sup(\phi)-\inf(\phi)<\ka_f$ (pressure gap), then $\phi$ admits a
unique equilibrium state $\mu_\phi$ on $J$. In this paper we prove
that the dynamical system ($f,\mu_\phi$) enjoys exponential 
decay of correlations of H\"older continuous observables as well as
the Central Limit Theorem and the Law of Iterated Logarithm for such
class of variables satisfying the natural co-boundary condition. We
also show that the topological pressure
function $t\mapsto P(t\phi)$ is real--analytic throughout the open set
of parameters $t$ for which  the potenials $t\phi$ have pressure gaps.
\end{abstract}

\maketitle

%***********************************************
\section{Introduction}

\sp\fr Fix an integer $k\ge 1$. Let $f:\Pk\to\Pk$ be a holomorphic
endomorphism of degree $d\ge 2$ of the complex projective space $\Pk$.
Denote by $J=J(f)$ the Julia set of the map $f:\Pk\to\Pk$, i. e.
the topological support of the measure of maximal entropy. The map
$f:\Pk\to\Pk$ is called regular if its exceptional set $E=E(f)$ does
not intersect the Julia set $J=J(f)$. Let $\phi:J(f)\to\R$ be a
continuous function, in the 
sequel frequently refered to as a potential. By $P(\phi)$  
we denote the (classical) topological pressure of the potential $\phi$
with respect to the dynamical system $f:J(f)\to J(f)$. Its definition
and a systematic account of properties can be found for example in
\cite{PU}.  If $\mu$ is a Borel probability $f$-invariant measure on
$J(f)$, we denote by $\hmu(f)$ its Kolmogorov--Sinai metric entropy. The
relation between pressure and entropy is given by the following celebrated
Variational Principle.
\begin{equation}\label{VP}
\P(\varphi)=\sup\lt\{\hmu(f)+\int\varphi d\mu\rt\},
\end{equation}
where the supremum is taken over all Borel probability $f$-invariant
measures $\mu$, or equivalently, over all Borel probability
$f$-invariant ergodic measures $\mu$. The measures $\mu$ for which 
$$
\hmu(f)+\int\phi\, d\mu=\P(\phi)
$$ 
are called equilibrium states for the potential
$\phi$. The main theorem proved in \cite{uzpk} was this.

\

\bthm\label{AA}
For every  regular holomorphic endomorphism $f:\Pk\to\Pk$ of a complex
projective space $\Pk$, $k\ge 1$, there exists a positive number
$\ka_f>0$ such that if $\phi:J(f)\to\R$ is a H\"older continuous
function with $\sup(\phi)-\inf(\phi)<\ka_f$ (we then say that $\phi$
has a pressure gap), then $\phi$ admits a
unique equilibrium state $\mu_\phi$ on $J$. This equilibrium state is
equivalent to a fixed point of the normalized dual Perron-Frobenius
operator. In addition the
dynamical system $(f,\mu_\phi)$ is K-mixing, whence ergodic. In the
case when the Julia set $J$ does not intersect any periodic
irreducible algebraic varieties contained in the critical set of $f$,
we have that $\ka_f=\log d$.
\ethm

\

\fr The main object of our paper will be the dynamical system
($f,\mu_\phi$). We shall show in Theorem~\ref{t220110627} that this
system enjoys exponential 
decay of correlations of H\"older continuous observables as well as
the Central Limit Theorem and the Law of Iterated Logarithm for such
class of variables satisfying the natural co-boundary condition. We
also show in Theorem~\ref{t1ai1} that the topological pressure
function $t\mapsto P(t\phi)$ 
is real--analytic throughout the open set of parameters $t$ for which
the potenials $t\phi$ have pressure gaps.

\sp\fr This paper is self-contained in the sense that all notions used
are introduced and all the steps leading to the main theorems are
explained. Majority of proofs however are exactly the same as those for the
$1$--dimensional case delt with in \cite{SUZ1}, and pointing out the
particular fragment of \cite{SUZ1} we refer the
reader to this paper for proofs.

\

\section{Good Holomorphic Inverse Branches}

\sp\fr Given an open connected subset $W$ of $\Pk$ and given an
integer $n\ge 1$, we denote by $I_n(W)$ the collection of all
connected components of $f^{-n}(W)$. If $V\in I_n(W)$ and $f|_V^n:V\to
W$ is a bijection (equivalently an injection), we set
$$
f_V^{-n}:=(f|_V^n)^{-1}:W\to V.
$$
We denote the callection of all such components by $\PG_n(W)$ and
refere to them as pre-good components. Of course, if $V\in\PG_n(W)$,
then the map $f_V^{-n}:W\to V$ is a holomorphic homeomorphism from $W$
to $V$. We call it the holomorphic inverse branch of $f^n$ from $W$ to
$V$. The main result of this section is the following.

\

\blem\label{l1fiw20110702}
For every $\g\in (0,1)$ and for every $\e>0$ there exist some integers
$l,q\ge 1$ and real numbers $\eta\in (0,1)$ and $\th>0$ such that if
$B$ is an arbitrary open ball centered at a point from the Julia set $J$, and if
$(1+\eta)^2B$ is disjoint from the set $\PC_l$, then for every integer $n\ge 1$
there exists a subset $G_n(B)\sbt \PG_{qn}(B)$ with the following properties. 

\begin{itemize}

\sp\item[(a)] For every $V\in G_n(B)$ there exists
              $W\in\PG_{qn}((1+\eta)B)$ such that $V\sbt W$.

\sp\item[(b)] If $V\in G_{n+1}(B)$, then $f^q(V)\in G_n(B)$.

\sp\item[(c)] If $V\in G_{n+1}(B)$, then $\diam(V)\le \g^n$.

\sp\item[(d)] $\mu_\phi\(\bu G_n(B)\)\ge (1-\e)\mu_\phi(B)$, where
              $\bu G_n(B):=\bu\{V:V\in G_n(B)\}$  
\sp\item[(e)] $\mu_\phi\(\bu B_n(B)\)\le e^{-\th qn}$, \nl
              where
              $B_n(B)$ consists of all connected components of the sets
              $f^{-qn}(V)$, $V\in G_{n-1}(B)$ that do not belong to 
              $G_n(B)$ and  $\bu G_n(B):=\bu\{V:V\in G_n(B)\}$. 
\end{itemize}
\elem

\

\fr Applying Cauchy's formulas for partial derivatives, we directly
obtain from Lemma~\ref{l1fiw20110702} the following.

\

\bcor\label{c120110706}
With the hypotheses of Lemma~\ref{l1fiw20110702} we have that there
exists a constant $C_\eta>0$ such that
$$
||Df_V^{-n}||_\infty\le C_\eta\g^n
$$
for all $n\ge 0$ and all $V\in G_n(B)$.
\ecor

\

\fr Let 
$$
\phi_q=\sum_{j=0}^{q-1}\phi\circ f^j
$$
The most signifcant consequence of belongin to $G_n(B)$ is this.
There exists a constant $C_q\ge 1$ such that for every $n\ge 1$, every
$V\in G_n(B)$ and all $x,y\in V$, we have that 
\begin{equation}\label{potentialdistortion}
\frac{\exp\(S_n\phi_q(x)\)}{\exp\(S_n\phi_q(y)\)}\le C_q,
\end{equation}
where in here $S_n$ refers to Birkhoff's sums corresponding to the
dynamical system $f^q:\Pk\to\Pk$.

\

\section{Selection of the Root Ball}

\sp\fr Simplyfying the proof of Lemma~9 in \cite{SUZ1}, we prove the
following.

\

\blem\label{l320110702}
Let $a<b$ be two real numbers. If $\mu$ is a Borel finite measure on
$[a,b]$, then for every $\l>1$ large enough there exist a point $c\in
(a,b)$ (in fact a measurable set of positive Lebesgue measure of such
points $c$) and $\tilde\l>1$ such that
$$
\mu\((c-\tilde\l^{-n},c+\tilde\l^{-n})\)\le \l^{-n}
$$
for all integers $n\ge 1$.
\elem

\sp\fr We now keep the setting of Lemma~\ref{l1fiw20110702} with $\e:=1/2$. As a
straightforward consequence of the abstract Lemma~\ref{l320110702}, we
shall prove the following.

\

\blem\label{l420110702}
For every point $w\in J\sms\PC_l$ there exists a ball $B$ centered at
$w$ such that $(1+\eta)B\cap\PC_l=\es$ and 
$$
\mu_\phi\(B(\bd B,\l^{-3n})\)\le\l^{-n}.
$$
\elem

\begin{proof}
Take any $R>0$ so small that $(1+\eta)B(w,R)\cap\PC_l=\es$ and consider
the map $P:\ov B(w,R)\to[0,R]$ given by the formula
$P(z)=||z-w||$. Applying Lemma~\ref{l320110702} to the measure
$\mu_\phi\circ P^{-1}$ the assertion of Lemma~\ref{l420110702} immediately
follows. 
\end{proof}

\

\section{Fine Inducing Scheme}\label{fis}

\sp\fr Assume without loss of generality that $\P(\phi)=0$. Fix an
integer $r\ge 1$ so large that
\beq\label{120110706}
C_\eta\g^r\le \frac14,
\eeq
where the constant $C_\eta$ comes from Corollary~\ref{c120110706}. More
requirements on $r$ will be imposed later. Put 
$$
h=f^{qr}:\Pk\to\Pk.
$$
and
$$
\phi_0:=\sum_{j=0}^{qr-1}\phi\circ f^j.
$$
Define
$$
\L_0:=\L_\phi^{qr}:C(J)\to C(J).
$$
So, $\L_0$ is the Perron-Frobenius operator associated to the map
$h:J\to J$ and potential $\phi_0:J\to\R$. It is given by the formula
$$
\L_0(g)(x)=\sum_{y\in h^{-1}(x)}g(y)\exp(\phi_0(y)).
$$
In exactly the same way as in \cite{uzpk} we can prove the following.

\

\begin{lem}\label{czest}
Assume that $Q\sbt\Pk$ is a set for which there exists $\beta>0$ such that 
\begin{equation}\label{frec}
\mathcal{L}_h(1_{Q})(x)>\beta
\end{equation}
for almost every $x\in J(f)$.
Then there exist $\alpha\in (0,1)$, an integer $n_0\ge 0$, and
$\delta>0$, all three depending on $\beta$ only (in particular
independent of $r\ge 1$), such that for all $n\ge n_0$ we have that,
\begin{equation}
\mu\lt(\lt\{x\in J(f):\#\left\{0\le i\le n: h^i(x)\in Q\right\} 
\le\alpha n\rt\}\rt)
<\exp(-\delta n).
\end{equation}
\end{lem}

\

\fr In order to make this lemma useable for us, we need the following.

\blem\label{l520110702}
For every integer $q\ge 1$ large enough there exists $\b>0$ such that
for every integer $r\ge 1$, we have for every $z\in J(f)$ that,
$$
\L_0(\1_Q)(z)\ge \b,
$$
where 
$$
Q:=Q_r:=\bu\{V:V\in G_{r-1}(B)\}.
$$
\elem

\begin{proof}
As a straightforward consequence of Lemma~\ref{l1fiw20110702},
particularly its item (c), and \eqref{potentialdistortion}, we get the
following.
\beq\label{1beta}
\L_\phi^{q(r-1)}\1_Q(z)\ge (2C_q)^{-1}
\eeq
for all $z\in B$. Now, invoking topological exactness of the dynamical
system $f:J(f)\to J(f)$, take any $q\ge 1$ so large that $f^q(B)=J(f)$. For
every $x\in J$ there then exists $y\in B$ such that
$f^q(y)=x$. Apllying \eqref{1beta} we can then write
$$
\aligned
\L_0(\1_Q)(x)
&\ge \exp\(S_q\phi(y)\)\L_\phi^{q(r-1)}\1_Q(y) \\
&\ge (2C_q)^{-1}\exp\(S_q\phi(y)\) \\
&\ge (2C_q)^{-1}\exp\(\inf\(S_q\phi\)\).
\endaligned
$$
Setting $\b$ to be the last number of this formula finishes the proof.
\end{proof}

\

\fr We say that a point $z\in B$ has a good pullback of length $n\ge
1$ if $h^n(z)\in B$ and the connected component of $h^{-n}(B)$
containg $z$ belongs to $G_{rn}(B)$. We frequently refer to this
component as a good pullback and denote it by $V_n(z)$. We further say
that such a good pullback $V_n(z)$ is very good if for every $0\le j\le n-1$,
$$
\dist\(h^j(V_n(z)),\bd B\)\ge \tilde\l^{n-j}.
$$
Note that if $r\ge 1$ is so large that 
$$
\g^r< \tilde\l^{-1},
$$
then every good pullback $V$ is entirely contained in $B$. Now, the
proof of Lemma~17 from \cite{SUZ1} goes verbatime in our present
setting and it asserts this. 

\

\blem\label{Yn}
As in Lemma~\ref{czest} let $Z_n$ be the set of all points $x\in B$ for which
$$
\# \left\{0\le i\le n:h^i(x)\in Q\right\}>\alpha n.
$$
Let
$$
Y_n
:=\lt\{x\in Z_n:\#\{0\le j\le n:{\rm the ~~pullback~~}V^x_j{\rm ~~is~~
     good}\}<(\alpha/2)n\rt\}.
$$
Then with $r\ge 1$ sufficiently large, we have that 
$$
\mu_\phi(Y_n)<e^{-{\alpha\over 8}nr\theta q}.
$$
\elem

\

\fr The proof of the next lemma is also the same as the proof of a lemmma
from \cite{SUZ1}. This time this is Lemma~18. The only modification is
that $\bd U$ is to be replaced by $\bd B$. 

\

\blem\label{verygood}
Let $R_n\subset B$ be the set of points in $x\in Z_n$ that satisfy the
following two requirements.

\begin{itemize}
\item[(1)]$x\in B\setminus Y_n$, i.e. the points in $R_n$ have
  at least ${\alpha\over 2}n$ good pullbacks, but 
\item[(2)] No good pullback $V_m(x)$ with $m\le n$, is very good. 
\end{itemize}
Then
$$
\mu_\phi(R_n)\le (\lambda'')^{-n}
$$
with some $\l''>1$ independent of $n$.
\elem

\

\section{Construction of the induced system}\label{constr}

\sp\fr Let 
$$
X=\bu_{n=1}^\infty Z_n\sms(Y_n\cup R_n)
$$
It directly follows from Lemma~\ref{czest}, Lemma~\ref{l520110702},
Lemma~\ref{Yn}, and Lemma~\ref{verygood} that 
\beq\label{muphiX=1}
\mu_\phi(X)=\mu_\phi(B).
\eeq
Given $x\in X$ let $n(x)$ be the smallest integer $n\ge 1$ such that
$x\in Z_n\sms(Y_n\cup R_n)$. We define
$$
F(x)=h^{n(x)}(x).
$$ 
Keep this $x\in X$ and put $n=n(x)$. Note that, if $y\in V_n(x)$
then this procedure, applied to $y$ leads to the same component
$V_n$. 
Indeed, by the definition of the induced map, we use the earliest very
good pullback. Thus, if $F(y)\neq h^n(y)$ then $F(y)=h^m(y)$ for some
$m<n$. Let $V_m(y)$ be the corresponding pullback. Then $V_m(y)\cap
V_n(x)\neq\emptyset$ as $y$ belongs to both of these sets, but
$V_n(x)\nsubseteq V_m(y)$ since $x\in V_n(x)\setminus V_m(y)$.  
 Let us consider $h^m( V_m(y))=U$ and $h^m(V_n(x))$. The latter is
an element of the pullback chosen for $x$ (a component of
$h^{-(n-m)}(B)$) and, since  $V_n(x)$ must intersect $\partial V_m(y)$, 
also $h^m(V_n(x))$ intersects $\partial B$. But this is impossible by
the definition of very good pullbacks. Let $\mathcal D$ 
be the countable family of all sets $V_{n(x)}(x)$, $x\in X$, defined in this
way. We have just shown that the function $n:X\to\N$ is constant on
each disc $D\in \mathcal D$, and so it can and will be treated as a
function from $\mathcal D\to\N$. In particular, the map  
$$
F:\bigcup_{D\in\mathcal D}\to B
$$
is well-defined and its inverse branches $F_D^{-1}:U\to D$,
$D\in\mathcal D$, form an infinite iterated function
systems, which, with a slight abuse of notation, will be also refered
to as $F$. We denote by $J_F$ the limit set of the iterated function
system $F$, i.e.
\beq\label{limisetIFS}
J_F=\bi_{n=0}^\infty F^{-n}(X).
\eeq
The argument leading to \eqref{muphiX=1} gives in fact more. Namely:
\beq\label{fullmeasurelimisetIFS}
\mu_\phi(J_F)=\mu_\phi(B).
\eeq    
It immediately follows from the construction of the system $F$ and
Lemma~\ref{verygood}, that
\begin{equation}\label{expdecay20110622}
m_\phi\lt(\bu\lt\{D:n(D)=n\rt\}\rt)\le (\lam'')^{-n}
\end{equation}
for all $n\ge 1$.
Let us record the following, proved in the same way as Lemma~19 in
\cite{SUZ1}, essential property of this induced system. 

\

\begin{lem}
If $D_1$, $D_2$ are two domains in $\mathcal D$, $F_{|D_1}=h^n$,
$F_{|D_2}=h^m$ then for $0\le s< n$, $0\le t< m$ either ${h^s(D_1)}\cap
{h^t(D_2)}=\emptyset$ or the closure of one 
of these sets is contained in the other set. 
\end{lem}

\

\fr For the sake of Proposition~\ref{muphiFinvariant}, we need to extend the
potential $\phi$ beyond the Julia sets $J(f)$. 

\

\begin{lem}\label{holderextension}
The function $\phi$ can be extended in a H\"older continuous manner,
with the same H\"older exponent, to the whole projective space $\Pk$.  
\end{lem}

\

\fr This lemma is well-known; a proof can be found in \cite{uzpk}.
From now on, we assume that the potential $\phi$ is defined and
H\"older continuous in the whole Riemann sphere $\oc$. 

\sp\fr As we have passed to induced system, we shall modify the potential
$\phi$ accordingly to this inducing process. 
First, if $D\in\mathcal{D}$ is one of discs on which $F$ is defined,
then we put, for all $x\in D$, that
$$
\hat\phi(x)=\sum_{k=0}^{n(D)-1}S_{qr}\phi(h^k(x)).
$$
Then, for all Borel sets $A\subset D_e$ we have that,
$$
\aligned
m_\phi(F(A))
&=m_\phi(h^{n(D)}(A))
=\int_A\exp\lt(-\sum_{k=0}^{n(D)-1}S_{qr}\phi\circ h^k\rt)dm_\phi \\
&=\int_A\exp\(-\hat\phi(x)\)dm_\phi(x).
\endaligned
$$
Along with \eqref{fullmeasurelimisetIFS} this entails the following.

\

\begin{lem}\label{mphiFconformality}
The probability measure $m_\phi$ is $\exp(-\hat\phi)$-conformal for the
map $F:J_F\to J_F$. 
\end{lem}

\

\fr Having Lemmas~\ref{mphiFconformality} and \ref{holderextension}, the
general theory of infinite iterated function systems, as developed 
in \cite{mu} along with \cite{MU}, gives the following.

\

\begin{prop}\label{muphiFinvariant}
There exists a unique probability $F$-invariant measure
$\mu_{\hat\phi}$ which is equivalent to $m_\phi$. Moreover the
Radon-Nikodym derivative $\hat\rho:={{d\mu_{\hat\phi}}\over{d m_\phi}}$
is bounded above and separated below from zero. This Radon-Nikodym
derivative $\hat\rho$ has a continuous extension $\hat\rho:B\to
(0;+\infty)$ to the whole ball $B$ and this extension is a fixed point
of the following transfer operator.
$$
\mathcal{L}_{\hat\phi}(v)(x)
=\sum_{y\in  F^{-1}(x)}\exp\hat\phi(y)v(y).
$$
This is a bounded linear operator acting on the Banach space $C_b(B)$
of all bounded real-valued continuous functions defined on $B$, and it is easy
to see that this operator is almost periodic. 
\end{prop}

\

\section{Real Analytyicity of Topological Pressure}

\sp\fr For every H\"older continuous potential $\phi:J(f)\to\mathbb{R}$, let
$$
\Delta_\phi=\lt\{t\in\mathbb{R}:\sup_{n\ge 1}\lt(
  P(t\phi)-\frac{1}{n}\sup (S_n(t\phi))\rt)>0\rt\}.
$$
Obviously, $\Delta_\psi$ is an open subset of $\mathbb{R}$. Having all
the material of the previous sections 
i.e. Section~\ref{fis} and Section~\ref{constr}, particularly 
formula \eqref{expdecay20110622}, we can repeat verbatime Section~6,
Real Analytyicity of the Pressure Function, from \cite{SUZ1} to get
the following. 

\

\begin{thm}\label{t1ai1}
The topological pressure function 
$$
\Delta_\phi\ni t\mapsto P(t\phi)\in\mathbb{R}
$$ 
is real--analytic.
\end{thm}

 
\section{Stochastic properties of the equilibrium
  measure $\mu_\phi$}\label{stochastic} 

\fr In this section we obtain strong transparent stochastic properties of the
dynamical system $(f,\mu_{\phi})$. We deduce them from the
corresponding properties of the induced system
$(F,\mu_{\hat\phi})$. We follow the scheme worked out in 
\cite{lsy} in the way it was presented in \cite{SUZ1}. We recall it briefly
now. We do this in an abstract 
context. Let $(\Delta_0,\mathcal{B}_0,m_0)$ be a
measure space with a finite measure $m_0$, let $\mathcal{P}_0$ be a
countable measurable  
partition of $\De_0$ and let $T_0:\Delta_0\to \Delta_0$ be a
 measurable map such that, for every 
  $\Delta'\in\mathcal{P}_0$ the map $T_0:\Delta'\to \Delta_0$ is a
  bijection onto $\Delta_0$. Moreover, we assume that   the partition
  $\mathcal{P}_0$ is generating, i.e. for every $x,y\in\Delta_0$ there
  exists $s\ge 0$ such that $T_0^s(x), T_0^s(y)$ are in different
  elements of the partition $\mathcal{P}_0$. We denote by $s=s(x,y)$
the smallest integer with this property and we
call it a separation time for the pair $x,y$.
We assume also  that for each $\Delta'\in\mathcal{P}_0$ the map
$({T_0}_{|\Delta'})^{-1}$ is measurable and that the Jacobian  
$Jac_{m_0}(T_0)$ with respect to the measure $m_0$ is well--defined and
positive a.e. in $\Delta'$. The 
following distortion property is assumed to be satisfied. 
\begin{equation}\label{youngcond0}
\left |\frac{\Jac_{m_0} T_0(x)}{\Jac_{m_0} T_0(y)}-1\right |
\le C\beta^{s_0(T_0(x),T_0(y))}.
\end{equation}
We have also a function $R:\Delta_0\to\mathbb{N}$ ("return time")
which is constant on each element of the partition  
$\mathcal{P}_0$.
We assume that the greatest common divisor of the values of $R$ is
equal to $1$. Finally, let 
$$
\Delta=\{(z,n)\in\Delta_0\times\mathbb{N}\cup\{0\}:0\le n<R(z)\}
$$
and each point $z\in\Delta_0$ is identified with $(z,0)$. The map $T$
acts on $\Delta$ as 
$$
T(z,n)=
\begin{cases}
(z, n+1)~~{\rm if}~~ n+1<R(z)\\
(T_0(z),0)~~{\rm if}~~ n+1=R(z)
\end{cases}
$$
The measure $m_0$ is spread over the whole space $\Delta$ by putting
$$
\tilde m_{|\Delta_0}=m_0 
\  \  \text{ and }  \  \
\tilde m_{|\Delta'\times\{j\}}
={m_0}_{|\Delta'}\circ\pi^{-1}_j, \  \De'\in \mathcal{P},
$$
where $\pi_j(z,0)=(z,j)$. Thus, the measure $\tilde m$ is finite iff
$\int_{\Delta_0}Rdm_0<\infty$. 
The separation time $s((x,n),(y,m))$ is defined to be equal to $s(x,y)$
if $n=m$ and $x,y$ are in the same set of the partition
$\mathcal{P}$. Otherwise we set $s(x,y)=0$. Given $\b>0$ we define
the space
$$
C_\beta(\De)=\{\varphi:\Delta\to \mathbb{R}:\exists \ C_\varphi~~{\rm such~
  that}~~|\varphi(x)-\varphi(y)|
<C_\varphi\beta^{s(x,y)}~~\forall x,y\in\Delta\}.
$$
We refer to the pentapol $\mathcal{Y}=(\De_0,T_0,\mathcal{P}_0,R,m_0)$
as a Young 
tower. The first three items of the following basic result have been
proved in \cite{lsy} while the fourth item was proved in \cite{SUZ1}.  

\

\begin{thm}\label{lsyoung0}
If $\mathcal{Y}=(\De_0,T_0,\mathcal{P}_0,R,m_0)$ is a Young tower and
$\int R dm_0<\infty$ then the following hold.

\sp\begin{enumerate}
\item{} There exists a unique probability  
$T$--invariant measure $\nu$, absolutely continuous with respect to
$\tilde m$. The Radon-Niokodym derivative $d\nu/d\tilde m$ is bounded
from below by a positive constant. The dynamical system $(T,\nu)$ is
exact, thus ergodic. 
\sp\item{} If $m_0(R>n)=O(\theta^n)$ for some $0<\theta<1$, then there
exists $0<\tilde\theta<1$ such that for all functions $\psi\in
L^\infty$ and we have $ g\in C_\beta$,  
\begin{equation}\label{cov}
Cov(\psi\circ T^n, g)
=\lt|\int (\psi\circ T^n) g d\nu-\int\psi d\nu\int g d\nu\rt|
=O(\tilde\theta^n) 
\end{equation}
\item{} If $m_0(R>n)=O(n^{-\alpha})$ with some $\alpha>1$ (in
  particular, if $m_0(R>n)=O(\theta^n)$), then  the Central Limit
  Theorem is satisfied for all functions $ g\in C_\beta$, that
  are not cohomologous to a constant in $L^2(\nu)$. 

\sp\item{} If $m_0(R>n)=O(n^{-\alpha})$ with some $\alpha>4$ (in
  particular, if $m_0(R>n)=O(\theta^n)$), then the Law of
  Iterated Logarithm holds for all functions $ g\in C_\beta$, that
  are not cohomologous to a constant in $L^2(\nu)$. This means that
  there exists a real positive number $A_g$  such that such that
  $\nu$ almost everywhere  
$$
\limsup_{n\to\infty}\frac{S_{n}g-n\int gd\mu}{\sqrt{n\log\log n}}=A_g. 
$$
\end{enumerate}
\end{thm}

\

\fr Passing to our holomorphic dynamical system ($f,\mu_\phi$) we
shall check that the assumptions of this  theorem are 
satisfied for our induced system $(F,m_\phi)$.    
The space $\Delta_0$ is now $J_F$, the limit set of the iterated
function system $F$. The partition $\mathcal{P}_0$ consists of the
sets $D\cap J_F$, $D\in\cD$. The measure $m_0$ is the
conformal measure $m_\phi$, restricted to $J_F$.  
The map $T_0$ is, in our setting, the map $F$. The function $R$, the
return time, is, naturally, defined as $R(D)=n(D)$. We shall check
that the pentapol $\cY_\phi=(J_F,F,\cP,n,m_\phi)$ is a Young
Tower, i.e. it satisfies the hypotheses of Theorem~\ref{lsyoung0}. We
start with the following. 

\

\blem\label{l220110705}
There exists a constant $C>0$ such that if $D\in\mathcal D$ and $x,y\in D$, then
$$
\dist(x,y)\le C4^{-s(x,y)}
$$
\elem

\begin{proof}
The assertion follows immediately from Corollary~\ref{c120110706},
formula \eqref{120110706}, and the definition of the separation time $s$. 
\end{proof}

\

\fr As a fairly straightforward consequence of this lemma, we get the
following.

\

\blem\label{l320110705}
If $D\in\cD$ and  and $x,y\in D$, then
$$
\dist(h^j(x),h^j(y))\le 2C2^{-s(x,y)}
$$
for all $0\le j\le n(D)$.
\elem

\begin{proof}
Note that $s(x,y)\ge 1$. Fix $0\le j\le n(D)$. It follows from
Corollary~\ref{c120110706} that
$\dist(h^j(x),h^j(y))\le\dist(h^{n(D)}(x),h^{n(D)}(y))$. But
$s(h^{n(D)}(x),h^{n(D)}(y))=s(x,y)-1$, it therefore follows from the 
previous lemma, that
$$
\dist(h^j(x),h^j(y))
\le\dist(h^{n(D)}(x),h^{n(D)}(y))
\le C2^{-s(h^{n(D)}(x),h^{n(D)}(y))}
=   2C2^{-s(x,y)}.
$$
We are done. 
\end{proof}

\

\fr We are now ready to prove the following.

\

\blem\label{l120110707}
The pentapol $\cY_\phi=(J_F,F,\cP,n,m_\phi)$ is a Young
Tower, i.e. it satisfies the hypotheses of Theorem~\ref{lsyoung0}. In
addition, $\tilde m_\phi(\De)<+\infty$.
\elem

\begin{proof}
First, we need to show that the formula \eqref{youngcond0}
holds. To do this fix an arbitrary disc $D\in\cD$ and aribtrary two
points $x,y\in J_F\cap D$. Recalling that the function $S_{qr}\phi:J(f)\to\R$ is 
H\"older continuous with some exponent $\a>0$, and using
Lemma~\ref{l320110705}, we can write as follows.
$$
\aligned
\bigg|\log\frac{\Jac_{m_\phi}F(y)}{\Jac_{m_\phi}F(y)}\bigg|
&=\bigg|\sum_{j=0}^{n(D)-1}S_{qr}\phi(h^j(y))
        -\sum_{j=0}^{n(D)-1}S_{qr}\phi(h^j(x))\bigg|\\    
&\le\sum_{j=0}^{n(D)-1}\bigg|S_{qr}\phi(h^j(y))-S_{qr}\phi(h^j(x))\bigg|\\
&\le\sum_{j=0}^{n(D)-1}C_1\dist^\a(h^j(y),h^j(x)) \\
&\le 2CC_1\sum_{j=0}^{n(D)-1}2^{-s(x,y)} 
=   2CC_1n(D)4^{-s(x,y)} \\
&\le 2CC_1s(x,y)4^{-s(x,y)} \\
&\le C_23^{-s(x,y)}
\endaligned
$$
with appropriate positive constants $C$, $C_1$, and $C_2$. 

\sp\fr We also need to take care of the last assumption in
Theorem~\ref{lsyoung0} requiring that the 
greatest common divisor of all the values of $n(D)$, $D\in\cD$, is
equal to $1$. If for our induced system this value is equal to some
integer $s>1$, then we replace the map $h$ by its iterate $h^s$. The
return times are now equal to $n(D)/s$, $D\in\cD$,
and their greatest common divisor equals $1$. 

\sp\fr The finiteness of $\tilde m_\phi(\De)$ follows
immediately from \eqref{expdecay20110622} and the definition of the
return time $R$.
\end{proof}

\

\fr Now consider $\pi:\De\to\Pk$, the natural
projection from the abstract tower $\De$ to the projective space $\Pk$
given by the formula
$$
\pi(z,n)=h^n(z).
$$
Then
\begin{equation}\label{eq120110623}
\pi\circ T =h\circ \pi,
\end{equation}
$$
{\tilde {m_\phi}}\big|_{J_F}\circ\pi^{-1}=m_0=m_\phi,
$$ 
and  
$$
\tilde{m_\phi}_{D\times\{n\}} \circ\pi^{-1} 
={m_\phi}_{D\times\{0\}}\circ h^{-n}
={m_0}_{|D}\circ h^{-n}
$$ 
for all $D\in\cD$ and all $0\le n\le n(D)$.
Now, the measure $\tilde{m_\phi}_{D\times\{n\}} \circ\pi^{-1}$ is
absolutely continuous 
with respect to $m_\phi$ with the Radon-Nikodym derivative equal to
$J_{D,n}:=\Jac_{m_\phi}(h^{-n})$ in $h^n(D)$ and zero elsewhere. Therefore,
$$
\aligned
\int_{\Pk}\sum_{D\in\cD}\sum_{n=0}^{n(D)-1}J_{D,n}\,dm_\phi
&=\sum_{D\in\cD}\sum_{n=0}^{n(D)-1}\int_{\Pk}J_{D,n}\,dm_\phi 
 =\sum_{D\in\cD}\sum_{n=0}^{n(D)-1}\int_{h^n(D)}J_{D,n}\,dm_\phi \\
&=\sum_{D\in\cD}\sum_{n=0}^{n(D)-1}\tilde{m_\phi}_{D\times\{n\}}
       \circ\pi^{-1}(h^n(D))\\     
&=\sum_{D\in\cD}\sum_{n=0}^{n(D)-1}\tilde{m_\phi}_{D\times\{n\}}
       \circ\pi^{-1}(\P^k)\\   
&=\tilde m_\phi\circ\pi^{-1}(\Pk) 
 =\tilde m_\phi(\De) \\
&<+\infty,
\endaligned
$$ 
where in writing the inequality sign we used the last assertion of
Lemma~\ref{l120110707}.
Thus, the function $\sum_{D\in\cD}\sum_{0\le n<n(D)}J_{D,n}$ is integrable
with respect to the measure $m_\phi$. This implies immediately that
the measure $\tilde m_\phi\circ\pi^{-1}$ is absolutely continuous with
respect to the measure $m_\phi$ with the Radon--Nikodyma derivative
equal to $\sum_{D\in\cD}\sum_{0\le n<n(D)}J_{D_i,n}$.  
Hence, the measure $\nu\circ\pi^{-1}$ is also absolutely continuous
with respect to $m_\phi$. Since $\nu$ is $F$--invariant and $\pi\circ
T =h\circ \pi$, the measure $\nu\circ\pi^{-1}$ is $h$--invariant. 
But the measure $\mu_\phi$ is $h$--invariant ergodic and
equivalent with the conformal measure $m_\phi$. Hence, $\nu\circ\pi^{-1}$
is absolutely continuous with respect to the ergodic
measure $\mu_\phi$. Innvariance and ergodicity of $\nu\circ\pi^{-1}$
yield this.  

\

\begin{lem}\label{l120110623}
$$
\nu\circ\pi^{-1}=\mu_\phi.
$$
\end{lem} 

\

\fr Having this, we can prove in exactly the same way as Theorem~56
in \cite{SUZ1}, the following. 

\

\begin{thm}\label{t220110627}
For the dynamical system $(f,\mu_\phi)$ the following hold.
\begin{enumerate}
\item{} For every $\alpha\le 1$, every $\alpha$--H\"older continuous
  function $g:J(f)\to \mathbb{R}$ and every bounded measurable
  function $\psi:J(f)\to\mathbb{R}$, we have that
$$
\bigg|\int\psi\circ f^n \cdot g d\mu_\phi
-\int g d\\mu_\phi\int\psi d\mu_\phi\bigg|
=O(\theta^n)
$$
for some $0<\theta<1$ depending on $\alpha$.

\sp\item{} The Central Limit Theorem holds for every H\"older continuous
  function $g:J(f)\to  \mathbb{R}$ that is not
  cohomologous to a constant in $L^2(\mu_\phi)$, i.e. for which there is no
  square integrable function $\eta$ for which $g={\rm
    const}+\eta\circ f-\eta$. Precisely this means that
 there exists $\sigma>0$ such that
$$
\frac{1}{\sqrt{n}}{\sum_{j=0}^{n-1}g\circ f^j} \to \mathcal N(0,\sigma)
$$ 
in distribution.
\sp\item{} The Law of Iterated Logarithm holds for every H\"older continuous
function $g:J(f)\to \mathbb{R}$ that is not cohomologous to a constant
in $L^2(\mu_\phi)$. This means that there exists a real positive
constant $A_g$ such that such that $\mu_\phi$ almost everywhere 
$$
\limsup_{n\to\infty}\frac{S_{n}g-n\int gd\mu}{\sqrt{n\log\log n}}=A_g. 
$$
\end{enumerate}
\end{thm}


















\begin{thebibliography}{9999999}

\bibitem[MU2]{mu} 
D. Mauldin, M.Urba\'nski, Graph directed Markov
  systems: geometry and dynamics of limit sets, Cambridge University
  Press, 2003.

\bibitem[MU3]{MU} 
D. Mauldin, M. Urba\'nski, On the uniqueness of the
  density of the invariant measure in an infinite hyperbolic iterated
  function system, Periodica Math. Hung. ,37 (1998), 47-53.

\bibitem[PU]{PU} 
F. Przytycki, M. Urba\'nski, Conformal Fractals ???-
Ergodic Theory Methods, Cambridge University Press, 2010.

\bibitem[SUZ]{SUZ1}
M. Szostakiewicz, M. Urba\'nski, A. Zdunik,
Fine Inducing and Equilibrium Measures for Rational Functions of the
Riemann Sphere, Preprint 2011.

\bibitem[LSY]{lsy} 
L. S. Young, Recurrence times and rates of
mixing, Israel Journal of Mathematics, 110 (1999), 153-188.

\bibitem[UZ]{uzpk}
M. Urba\'nski, A. Zdunik,Equilibrium Measures for Holomorphic
Endomorphisms of Complex Projective Space, Preprint 2009.


\end{thebibliography}
\end{document}












